\section{Desarrollo}

Tomando como punto de partida el c\'odigo suministrado por la cátedra, introdujimos las siguientes modificaciones para agregar dos funcionalidades: por un lado, la introducción de Delay y por el otro, la posibilidad de definir una probabilidad \textit{p} de pérdida de paquetes con valores que se ubiquen entre el 0 y el 1 (donde 0 no pierde ningún paquete y 1 lo pierde todos). Estos mismos seran efectuados sobre el envio de paquetes de acknowledgment $ACK$. Provocando que quien envia el paquete, crea que este no fue entregado correctamente.

La implementación se hizo de tal manera que tanto el valor del tiempo de delay como la probabilidad de la p\'erdida de paquetes se setean como par\'ametros al ejecutar la funci\'on.

Con el fin de agregar estas funcionalidades, se realizaron los siguientes cambios al c\'odigo original:

\begin{itemize}

\item Dentro del archivo \textbf{handler.py}:

\begin{itemize}

\item Agregamos la funci\'on \textit{se\_perdio\_paquete} que toma la probabilidad pasada por parámetro y decide sobre la base de esa probabilidad si el paquete que va a enviar efectivamente se va a perder o no.

\item Dentro de la función \textit{send\_ack} agregamos un condicional que llama a la función "se\_perdio\_paquete".
Si esta decide que el paquete se perdió, el condicional nos lleva a una sentencia de return y el paquete no se envía.
Luego, agregamos también dentro de esta función un Delay simulado mediante la instrucción $time.sleep()$ y cuyo valor es un porcentaje, pasado por par\'ametro, del delay m\'aximo permitido por el protocolo (1 segundo). Para enviar un paquete, la función espera la cantidad de tiempo indicada antes de ejecutar el \textit{build\_packet} y el \textit{send}.

\end{itemize}
\end{itemize}

Luego de algunas pruebas con estas implementaciones, pudimos observar que aun cuando estabamos perdiendo todos los $ACK's$, el protocolo funcionaba correctamente, asi decidimos que era una buena idea agregar tambien, la p\'erdida a la actualizaci\'on de la ventana para realizar nuevas pruebas.

\begin{itemize}

\item Dentro del archivo \textbf{protocol.py}:

\begin{itemize}

\item Agregamos la funci\'on \textit{se\_perdio\_paquete} que se comporta de manera similar a la mencionada en el item anterior dentro del archivo handler.py

\item Dentro de la función receive se agrega un condicional que llama a la funci\'on \textit{se\_perdio\_paquete} mencionada en el item anterior. El condicional enviar\'a la actualizaci\'on de ventana unicamente si esta funci\'on decide que el paquete a enviar no se va a perder. Si la funci\'on retorna que el paquete debe perderse, no se env\'ia la actualizaci\'on de ventana y se imprime por consola que el paquete se perdi\'o.

\item Adem\'as, en la funci\on $retransmit_packets_if_needed()$ se agrego un contador para las retransmiciones, para poder evaluar la cantidad de retransmiciones en funci\'on del delay.

\end{itemize}

\end{itemize}
 