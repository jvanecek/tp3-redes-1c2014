\section{Desarrollo}

El objetivo del TP es simular el delay y la p\'erdida de paquetes en una red real y poder experimentar con ellos. Asique tenemos que de alguna forma modificar el protocolo \emph{PTC} para simular estos fen\'omenos. 

Ambas situaciones se dan al momento que se env\'ian datos, que hay un retraso de $t$ unidades de tiempo y que dada una probabilidad $p$ este paquete se pierde o no. Por tanto, modificaremos la funci\'on \code{send\_ack()} de la clase \code{IncomingPacketHandler} para que tenga en cuenta a $t$ y $p$. 

Puntualmente, agregamos a \code{IncomingPacketHandler} dos variable: \code{porcentaje\_delay} de un tiempo de delay m\'aximo para calcular $t$ y \code{porcentaje\_perdida} que representa a $p$; y dos m\'etodos: 

\begin{verbatim}
def delay(self): 
  time.sleep(self.porcentaje_delay*MAX_DELAY)
\end{verbatim}

\begin{verbatim}
def se_perdio_paquete(self):
  valores = (1,0) # (True, False)
  proba = (self.porcentaje_perdida, 1-self.porcentaje_perdida)
  custm = stats.rv_discrete(name="custm",values=(valores, proba))
  return (custm.rvs(size=1) == 1)
\end{verbatim}

Estas dos funciones nos permiten simular lo que quer\'iamos. Con la constante \code{MAX\_DELAY} (que arbitrariamente la seteamos en el doble de \code{RETRANSMISSION\_TIMEOUT}, porque es un tiempo que nunca se va a superar) y un porcentaje de este valor elegido por el usuario, la funci\'on \code{delay()} duerme el thread esa cantidad de tiempo. 

El segundo m\'etodo usa el m\'odulo stats\footnote{\url{http://docs.scipy.org/doc/scipy/reference/tutorial/stats.html}} de la librer\'ia SciPy\footnote{\url{http://www.scipy.org/}} para determinar si el paquete se pierde. Se crea primero una funci\'on distribuci\'on discreta $custm$ tal que: 

\begin{eqnarray*}
 custm(\text{``Se pierde el paquete''}) &=& \texttt{porcentaje\_perdida} \\
 custm(\text{``No se pierde''}) &=& 1 - \texttt{porcentaje\_perdida}
\end{eqnarray*}

Luego, la funci\'on \code{rvs(size=1)} genera un valor aleatorio con la distribuci\'on creada.  

Como nuestra \'ultima modificacion a \emph{PTC}, usamos estas dos funciones en \texttt{send\_ack()} la que termina quedando de la siguiente manera:

\begin{verbatim}
def send_ack(self):
  if self.se_perdio_paquete():
    return # se perdio el ack asique no manda datos

  # simulacion de delay
  self.delay()

  ack_packet = self.build_packet()
  self.socket.send(ack_packet)
\end{verbatim}

% ?PORQUE LO TERMINAMOS PONIENDO ACA? DEBERIAMOS EXPLICAR PORQUE NO TERMINAMOS SACANDO LO DEL UPDATE WINDOW, QUE SE COLGABA. 