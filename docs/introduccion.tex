\section{Introducci\'on}

La capa de transporte, es la capa de nivel 4 en el modelo OSI, es la capa encargada del transporte de datos de una maquina a otra. Para permitir este intercambio de informaci\'on y tener la certeza de que la informaci\'on a sido bien entregada, manteniendo su integridad, existen distintos protocolos. Entre ellos, el mas utilizado es TCP.
TCP, nos asegura el env\'io sin perdida de informaci\'on, el orden de los datos y la confiabilidad.
Para poder cumplir con su objetivo el protocolo debe resolver ciertos problemas que se le presentan en la red, entre ellos nos encontramos con la demora o $delay$ y la $perdida de paquetes$. Estos pueden producirse por diversas causas en la red, y son disparadores de acciones en nuestro protocolo.

El objetivo del trabajo realizado, es poder comprender el comportamiento de la capa de transporte. Para ello, utilizaremos un protocolo de transporte implementado por la c\'atedra de la materia, sobre el cual efectuaremos ciertos cambios que nos permitan simular los problemas presentes en una red. Pudiendo asi, controlar estas variables, y sacar conclusiones sobre los efectos que estas tienen sobre el comportamiento del protocolo.
