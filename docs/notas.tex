Consignas del TP

1) MODIFICACION DEL PROTOCOLO
	1.1) Introducir delay al momento de enviar los ACKs. Este valor puede ser constante a lo largo del ciclo de vida de una instancia del protocolo, pero no obstante debe poder ser fácilmente editable para poder realizar el análisis de la segunda consigna.

	1.2) Definir una probabilidad p de pérdida de paquetes. Una forma posible de simularla es, al momento de enviar un ACK, decidir si éste será efectivamente enviado “tirando una moneda” con dicha probabilidad.

FORMA DE HACERLO: 
	A) Desde thread.PacketSender
	B) Desde protocol.PTCProtocol.attempt_to_send_data
	C) Desde protocol.PTCProtocol.send_and_queue
	D) Desde soquete.Soquete.send


2) EXPERIMENTACION
Para el protocolo original y para las versiones con ACKs demorados/perdidos, en el contexto de una red local (LAN), enviar archivos de distinto tamaño (e.g., 1 KB, 5 KB, 10 KB, 50 KB, 100 KB, etc.) y medir el tiempo total de transmisión de los mismos. Tomar la medida representativa como el promedio de un número N de experimentos. A partir de esto, analizar cómo impactan los efectos de red simulados en la performance del protocolo. Por ejemplo, graficar lo siguiente y sacar conclusiones: 

	- Throughput percibido (i.e., cantidad de bits por segundo) en función del tamaño de archivo.
	- Throughput en función del delay en los ACKs (para un tamaño de archivo constante).
	- Cantidad de retransmisiones en función del delay en los ACKs (para un tamaño de archivo constante).

Notar que esta lista no es exhaustiva. Se valorará especialmente la creatividad a la hora de presentar y
analizar los resultados obtenidos.




=============================
StackTrace

ptc_socket.py 	Socket.send(data)
protocol.py 	PTCProtocol.send(data)
thread.py		PacketSender.notify()
protocol.py 	PTCProtocol.handle_outgoing()
protocol.py		PTCProtocol.attempt_to_send_data():
protocol.py 	PTCProtocol.send_and_queue(packet)
soquete.py 		Soquete.send(packet)
