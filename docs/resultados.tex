\section{Resultados}


A medida que fuimos desarrollando nuestros experimentos, y modificando nuestro c\'odigo, nos hemos encontrado con distintas caracteristicas, algunas de las cuales queremos mostrar a traves de los siguientes gr\'aficos.

A continuaci\'on presentamos un gr\'afico del Throughput en funci\'on del tama\~no del mensaje, para distintos delays.

\ponerGrafico{./graficos/throughtput_vs_tamanio.png}{caption}{0.5}{label}
Buffer 1024 bytes.

Aqui podemos observar, como disminuye el throughput a medida que aumenta el delay. Tambien podemos observar que el throughput aumenta a medida que aumenta el tama\~no del mensaje. Esto se debe, a el aprovechamiento del buffer y queda evidenciado en los saltos que se hacen presentes cada 1024 Bytes, donde delimita el uso completo del buffer.
Por ende, vemos como aumenta la eficiencia cuando los mensajes aprovechan el tama\~no total de la ventana.


Ahora, presentaremos 4 gr\'aficos correspondientes al tama\~no del mensaje en funci\'on del tiempo, para distintos tama\~nos de buffer, comparando los resultados para distintos delays y porcentajes de perdida de paquetes.

\ponerGrafico{./graficos/size_vs_tiempo_b500_n9_[perdida].png}{caption}{0.3}{label}
Tiene el max buffer seteado en 500 Bytes. Se evalu\'o el tiempo que tarda el protocolo en funci\'on del tama\~nos, y distintos porcentajes de perdidas. El delay esta seteado en 0. 

\ponerGrafico{./graficos/size_vs_tiempo_b1024_n49_[perdida].png}{caption}{0.3}{label}
El mismo que el anterior pero con un buffer de 1024, y una mayor granularidad en los tama\~nos.

Viendo los resultados obtenidos, para distintos porcentajes de perdida en dos buffers distintos. Vemos que no difieren demasiado los resultados obtenidos. Debido a la aleatoreidad de la perdida de paquetes nos encontramos con algunos picos que parecen indicarnos que se han producido una cantidad de perdidas considerables en algunas pruebas. 

Sin embargo, sabemos que el echo de la actualizaci\'on de ventana, seguramente este afectando nuestro resultados. Para evitar estos, nos dispusimos a agregar perdida de paquetes no solo a los $ACK's$, sino, tambi\'en las actualizaciones de ventana.
Era nuestra intenci\'on presentar los gr\'aficos correspondientes a los mismos experimentos con dichos cambios al c\'odigo, pero nos encontramos con problemas a la hora de ejecutarlo.


\ponerGrafico{./graficos/size_vs_tiempo_b1024_n9_[delay].png}{caption}{0.3}{label}
Tiene el max buffer seteado en 1024 Bytes. Se evalua\'o el tiempo que tarda el protocolo en funci\'on del tama\~nos, y distintos porcentajes de delay. La perdida esta fijada en 0. 

\ponerGrafico{./graficos/size_vs_tiempo_b1024_n49_[delay].png}{caption}{0.3}{label}
El mismo que el anterior pero m\'as granularidad en los tama\~nos evaluados.

En este caso, podemos ver como aumenta el tiempo de transmision a medida que aumenta el delay, tambi\'en se pueden observar  esos saltos correspondientes al tama\~no del buffer. Los gr\'aficos parecen bastante explicitos, y no dejan mucho para decir. A medida que aumenta el tama\~no del mensaje aumenta el tiempo de transmisi\'on, este tiempo presenta un aumento marcado cuando se llena el buffer y debe utilizarse la primitiva de recepcion de datos mas de una vez.


\ponerGrafico{./graficos/delay_vs_tiempo.png}{caption}{0.3}{label}
Buffer 1024 bytes. 

Como era de esperar, mientras mas delay el tiempo aumenta, en este gr\'afico vemos como para los mismos tama\~nos de mensaje el tiempo aumenta con el delay. 


\ponerGrafico{./graficos/perdida_vs_tiempo.png}{caption}{0.3}{label}
Buffer 1024 bytes.

En cierto punto, estos resultados son los esperados. Los mensajes de mayor tama\~no son los que mas tiempo tardan. Sin embargo, esperariamos que a medida que aumenta el porcentaje de perdida tambi\'en aumente el tiempo.
Como mencionamos anteriormente, esto seguramente se deba a la actualizaci\'on de ventana, la cual permite al protocolo continuar transmitiendo los siguientes paquetes, aun sin haber recibido los $ACK's$ correspondientes.