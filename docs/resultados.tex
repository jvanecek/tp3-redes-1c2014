\section{Experimentaci\'on}

Para evaluar el impacto del delay y las perdidas de paquetes dise\~namos un set de experimentos para estudiar la variaci\'on del RTT ante los cambios de estas variables. 

Para correr los experimentos creamos un cliente y un server que intercambian paquetes. Una vez establecida la conexi\'on, el server le avisa al cliente que ya puede mandar los datos, y ah\'i es cuando el server empieza a medir el tiempo hasta que recibe el \'ultimo byte del cliente. 

A partir de este modelo, hay que tener en cuenta que el tama\~no del buffer del server va a influir en el RTT final, ya que cu\'anto m\'as chico, el cliente necesita mandar m\'as paquetes para mandar la misma cantidad de bytes. 

Los siguientes experimentos los corrimos con un cliente y un server que est\'an en el mismo host, para que las mediciones no est\'en viciadas por factores de una LAN real. 

\begin{enumerate}
\item Bytes enviados vs Throughput. Delay variable. Perdida nula. Buffer de 1024. 

  A continuaci\'on presentamos un gr\'afico del Throughput en funci\'on del tama\~no del mensaje, para distintos delays.

  \ponerGrafico{./graficos/throughtput_vs_tamanio.png}{caption}{0.5}{label}
  Buffer 1024 bytes.

  Aqui podemos observar, como disminuye el throughput a medida que aumenta el delay. Tambien podemos observar que el throughput aumenta a medida que aumenta el tama\~no del mensaje. Esto se debe, a el aprovechamiento del buffer y queda evidenciado en los saltos que se hacen presentes cada 1024 Bytes, donde delimita el uso completo del buffer.
  Por ende, vemos como aumenta la eficiencia cuando los mensajes aprovechan el tama\~no total de la ventana.


\item Bytes enviados vs Tiempo. Delay nulo. Perdida variable. Buffer de 500. Poca granularidad de los bytes

  \ponerGrafico{./graficos/size_vs_tiempo_b500_n9_[perdida].png}{caption}{0.3}{label}
  Tiene el max buffer seteado en 500 Bytes. Se evalu\'o el tiempo que tarda el protocolo en funci\'on del tama\~nos, y distintos porcentajes de perdidas. El delay esta seteado en 0. 

\item Bytes enviados vs Tiempo. Delay nulo. Perdida variable. Buffer de 1024. Mucha granularidad.  

  \ponerGrafico{./graficos/size_vs_tiempo_b1024_n49_[perdida].png}{caption}{0.3}{label}
  El mismo que el anterior pero con un buffer de 1024, y una mayor granularidad en los tama\~nos.

  Viendo los resultados obtenidos, para distintos porcentajes de perdida en dos buffers distintos. Vemos que no difieren demasiado los resultados obtenidos. Debido a la aleatoreidad de la perdida de paquetes nos encontramos con algunos picos que parecen indicarnos que se han producido una cantidad de perdidas considerables en algunas pruebas. 

\item Bytes enviados vs TIempo. Delay variable. Perdida nula. Buffer de 1024. Poca granularidad de los bytes.

  \ponerGrafico{./graficos/size_vs_tiempo_b1024_n9_[delay].png}{caption}{0.3}{label}
  Tiene el max buffer seteado en 1024 Bytes. Se evalua\'o el tiempo que tarda el protocolo en funci\'on del tama\~nos, y distintos porcentajes de delay. La perdida esta fijada en 0. 

  \ponerGrafico{./graficos/size_vs_tiempo_b1024_n49_[delay].png}{caption}{0.3}{label}
  El mismo que el anterior pero m\'as granularidad en los tama\~nos evaluados.

  En este caso, podemos ver como aumenta el tiempo de transmision a medida que aumenta el delay, tambi\'en se pueden observar  esos saltos correspondientes al tama\~no del buffer. Los gr\'aficos parecen bastante explicitos, y no dejan mucho para decir. A medida que aumenta el tama\~no del mensaje aumenta el tiempo de transmisi\'on, este tiempo presenta un aumento marcado cuando se llena el buffer y debe utilizarse la primitiva de recepcion de datos mas de una vez.


\item Delay vs Tiempo. Tama\~no variable. Perdida nula. Buffer de 1024.

  \ponerGrafico{./graficos/delay_vs_tiempo.png}{caption}{0.3}{label}
  Buffer 1024 bytes. 

  Como era de esperar, mientras mas delay el tiempo aumenta, en este gr\'afico vemos como para los mismos tama\~nos de mensaje el tiempo aumenta con el delay. 

\item Perdida vs Tiempo. Tama\~no variable. Delay nulo. Buffer de 1024.
  \ponerGrafico{./graficos/perdida_vs_tiempo.png}{caption}{0.3}{label}
  Buffer 1024 bytes.

\end{enumerate}