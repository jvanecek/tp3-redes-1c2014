\section{Experimentaci\'on}

Para evaluar el impacto del delay y las perdidas de paquetes dise\~namos un set de experimentos para estudiar la variaci\'on del RTT ante los cambios de estas variables. 

Para correr los experimentos creamos un cliente y un server que intercambian paquetes. Una vez establecida la conexi\'on, el server le avisa al cliente que ya puede mandar los datos, y ah\'i es cuando el server empieza a medir el tiempo hasta que recibe el \'ultimo byte del cliente. 

A partir de este modelo, hay que tener en cuenta que el tama\~no del buffer del server va a influir en el RTT final, ya que cu\'anto m\'as chico, el cliente necesita mandar m\'as paquetes para mandar la misma cantidad de bytes. 

Los siguientes experimentos los corrimos con un cliente y un server que est\'an en el mismo host, para que las mediciones no est\'en viciadas por factores de una LAN real.

\subsection{Bytes enviados vs Tiempo}
\subsubsection{Experimento A (Impacto de la p\'erdida)}

  Para evaluar el impacto de la p\'erdida de paquetes, creamos una conexi\'on que no tuviera delay y obtuvimos los siguientes gr\'aficos: 
  
  \ponerGrafico{./graficos/size_vs_tiempo_b512_n36_[perdida].png}{Buffer de 512 Bytes}{0.3}{size_tiempo_perdida_512}

  \ponerGrafico{./graficos/size_vs_tiempo_b1024_n49_[perdida].png}{Buffer de 1024 Bytes}{0.3}{size_tiempo_perdida_1024}

  En ambos gr\'aficos vemos como en cada m\'ultiplo del buffer las curvas tienen un salto, pero sin embargo parece que los distintos porcentajes de p\'erdidas no influyen en el tiempo, ya que los RTT son similares. 
  
\subsubsection{Experimento B (Impacto del delay)} 

  \ponerGrafico{./graficos/size_vs_tiempo_b512_n36_[delay].png}{Buffer de 512 Bytes}{0.3}{size_tiempo_delay_512}
  
  \ponerGrafico{./graficos/size_vs_tiempo_b1024_n49_[delay].png}{Buffer de 1024 Bytes}{0.3}{size_tiempo_delay_1024}
  
  En este caso, otra vez se mantiene los saltos de tiempos cuando la cantidad de bytes supera un m\'ultiplo del buffer. Pero a diferencia del Experimento A, el cambio de delay s\'i afecta al RTT. 

\subsection{Delay vs Tiempo}
\subsubsection{Experimento A}

  El siguiente gr\'afico lo hicimos usando una conexi\'on sin p\'erdida de paquetes y cuyo server usaba un B\'uffer de 1024 bytes.

  \ponerGrafico{./graficos/delay_vs_tiempo.png}{caption}{0.3}{label}

  Paralelamente a los gr\'aficos \ref{fig:size_tiempo_delay_512} y \ref{fig:size_tiempo_delay_1024} el tiempo crece en funci\'on del delay, y las curvas de los datos que usan las mismas r\'afagas de 1024 bytes para llegar del cliente al server tienen el mismo delay. En este caso, 500 bytes y 1000 bytes por un lado, por otro 1500 B y 2000 B, y por \'ultimo 2500 bytes. 

\subsection{P\'erdida vs Tiempo}
\subsubsection{Experimento A}
  Tama\~no variable. Delay nulo. Buffer de 1024.
  
  \ponerGrafico{./graficos/perdida_vs_tiempo.png}{caption}{0.3}{label}

\subsection{Bytes enviados vs Throughput}
  Delay variable. P\'erdida nula. Buffer de 1024. 

  \ponerGrafico{./graficos/throughtput_vs_tamanio.png}{caption}{0.5}{label}

  Aqu\'i podemos observar, como disminuye el throughput a medida que aumenta el delay. Tambi\'en podemos observar que el throughput aumenta a medida que aumenta el tama\~no del mensaje. Esto se debe a que el aprovechamiento del buffer y queda evidenciado en los saltos que se hacen presentes cada 1024 Bytes, donde delimita el uso completo del buffer.
  Por ende, vemos como aumenta la eficiencia cuando los mensajes aprovechan el tama\~no total de la ventana.
